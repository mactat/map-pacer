In the future, it would be beneficial to expand the scope of this master thesis by implementing more algorithms to address different use cases and lower the complexity, especially for multi-agents path planning. Additionally, improving the robustness of the current algorithms should be a priority. This could involve testing the algorithms under different conditions, such as high load or network latency, and implementing failover mechanisms to ensure that the algorithms continue to function properly.

Another area for future work is to test different communication protocols in addition to MQTT. This could include protocols such as HTTP or AMQP, and could help to identify the most suitable protocol for different use cases. Experimenting with different Quality of Service (QoS) settings could help to optimize the system's performance and ensure that it meets all requirements.

To further improve the software's multi-tenancy capabilities, it would be beneficial to implement more robust mechanisms for ensuring that different tenants' data is isolated from one another. This could involve implementing encryption or access controls at different levels of the system, as well as logging the system.

In addition to these improvements, it would also be interesting to experiment with the kube-edge software to see how it can be integrated into the system. This could provide additional scalability and fault tolerance capabilities. It can also lead to better user experience as agents' machines would be managed from the cloud.

Overall, these future improvements could help to make the software more robust and widely applicable, and it could be distributed as a Software as a Service (SaaS) software to make it easier for organizations to adopt and use.