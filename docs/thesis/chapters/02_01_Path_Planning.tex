%!TEX root = ../Thesis.tex
Path planning is the process of finding a correct path from source to destination with a focus on avoiding obstacles, for example, navigating a mobile robot in a factory setup from one place in a factory to another\cite{path_planning}. In mobile robotics, the environment in which the robot operates is often represented by a map. The map can be represented in either 2D or 3D, depending on the environment's complexity and the robot's capabilities. 2D maps are simpler to work with and are often used for robots that operate in relatively flat environments. On the other hand, 3D maps are used for robots operating in more complex environments, such as buildings or mines.

In terms of path planning, map geometry can differ using different types of algorithms. Some popular algorithms include Artificial Potential Field (APF), Sampling-based methods, and grid-search algorithms. Grid-search algorithms use a grid to represent the environment, and they work by searching for a path through the grid that avoids obstacles and leads the robot to the goal. These algorithms are simple to implement and are computationally efficient \cite{not_grid_based}. This master thesis focuses on 2D path planning and  puts emphasis on grid-search algorithms as opposed to different solutions such as Artificial Potential Field or Sampling-Based algorithms, as it is simple, computationally efficient, and well-suited for mobile robots with limited resources.

Grid-based algorithms assume that the world can be represented as a grid and every position of an agent can be described as a grid point. The robot, which also will be referred to as an agent is only allowed to move to adjacent grid points, so the available state space is discretized which simplifies the computation. The drawback is that solutions may be sub-optimal in relation to the real-world scenario as only a limited number of point is being considered \cite{SARANYA2014766}.

\subsection{KEPP vs UEPP}
Path planning problems can be classified into two categories: known-environment path planning (KEPP) and unknown-environment path planning (UEPP). The former category assumes that there is a complete knowledge of the environment prior to the execution of the algorithm, and there is no need for an exploration phase or sensing. The latter category incorporates sensing information into the planning process, thus, complete and absolute information is not required during the execution of the algorithms. While KEPP is more straightforward to execute, UEPP more closely resembles real-world scenarios. In this study, only KEPP solutions will be discussed, as the primary objective is to effectively address multi-agent system path planning and information gathering and environment sensing are not essential for this process \cite{path_planning_protocols}.