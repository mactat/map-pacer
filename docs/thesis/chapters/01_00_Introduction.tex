%!TEX root = ../Thesis.tex
\chapter{Introduction}

Path planning is a process of finding a route between starting position and a goal. It is a common task in the current industry setup, used for autonomous navigation. Navigation involves three main challenges: determining the current location (localization), deciding on a route to follow (path planning), and controlling the movement of the robot (motion control). Of these, path planning is considered to be particularly significant because it involves choosing an appropriate path for the robot to take within its environment \cite{path_lanning_intro}. 

Mobile robots have become increasingly important in today's world due to their ability to perform tasks that are either dangerous or difficult for humans to do. Path planning is a crucial aspect of mobile robotics, as it enables robots to navigate to a desired destination while avoiding obstacles and other hazards. This is particularly important in fields such as manufacturing, entertainment, medicine, mining, rescue, education, military, space, and agriculture where robots can help to increase efficiency and safety. For example, in manufacturing, robots can work in hazardous environments such as nuclear power plants or mines without putting human lives at risk. In the same environments, mobile robots usually are more performant than their human counterpart. In rescue, robots can be used to search for survivors in dangerous environments such as collapsed buildings. In addition, mobile robots with advanced path-planning capabilities can be used to make the delivery and transportation of goods more efficient and cost-effective. Furthermore, Mobile robots, specifically Automated Guided Vehicles (AGVs), have found wide adoption in the manufacturing and logistics industries, where they are used to transport materials and goods within a facility. They are particularly useful in large manufacturing plants or warehouses where a lot of material handling is required \cite{intro_mobile_robots}. 

While more mobile robots are being used it becomes important to solve this task efficiently using state-of-the-art approaches. As path planning is computationally intensive in a large setup, it might require more resources than a single mobile robot (agent) can provide, therefore, this project will explore collaborative computing between the agents as well as edge-cloud collaborative computing. The focus is not only on the implementation of the algorithms but rather on an overall end-to-end system that can be used both with simulated agents as well as real edge devices.

In the second chapter theory behind multi-agent path planning is introduced along with an explanation of standard algorithms designed for this purpose. It also serves as an introduction to the cloud and edge computing as both of those technologies are used in this project. Finally, this section elaborates on standard communication protocols and algorithms used in agent implementation.

The third chapter focuses on the design of the system, with an emphasis on creating a system that can be easily simulated but also embedded in the real robot or stand-alone computing unit. It elaborates on specific communication patterns among the agents and the general architecture of both local/edge and cloud entities.

The fourth chapter explains in detail how the system was implemented. It starts with a description of the development process and goes deeper into more specific technologies and their placement in the project. It also shows the final look of the system as well as its user interface of it. The system testing approach is explained in the fifth chapter, focusing on reproducible test cases along with test results with different algorithms and resource profiles.

The sixth chapter serves as a summary of the project and explains what can be improved in future work. The project plan was created beforehand and it is attached in Appendix \ref{sec:app_06}. 

\section{Problem statement}
The problem that this master thesis aims to address is how to optimize the performance of path-planning algorithms in a distributed system. Specifically, the research aims to determine whether it is more beneficial to run path planning algorithms in the cloud or locally, and which algorithms are most suitable for different use cases. Moreover, this thesis tries to investigate the impact of network latency on algorithm performance and to determine the limits of agent resources that make it more beneficial to run path planning algorithms in the cloud rather than locally. The focus of this master thesis is not just on comparing different path-planning algorithms, but also on the implementation of a distributed system that can effectively utilize these algorithms. The ultimate goal of this research is to provide a robust and scalable system that can meet the needs of a wide range of use cases.

\section{Robots fleet in smart factories}
One of the main use cases of the proposed system can be robot fleets in factories. Mobile robots are widely used in production plants, but the most popular solution is to use AGV - an automated guided vehicle \cite{agv}. The difference between AGV and the fully autonomous mobile robot is that AGV will usually follow the guided path marked by the magnetic or traditional marker. This solution allows to simplify robot setup, by reducing the number of sensors, but at the same time, it is less flexible and requires constant maintenance of marked paths. The use of AGVs with advanced path planning capabilities can help to improve the efficiency and cost-effectiveness of the transportation and delivery of goods within a facility. This can lead to increased productivity, reduced labor costs, improved inventory management, and flexibility in adapting to changes in the production schedule 
 \cite{a_star_factories}.

Using multi-agent path planning, robots are not required to follow any marked path. However, they need to have more sensors to locate each other in the factory. This solution enables robots to reach any point in the factory and not collide with each other. It is also very extensible as any number of robots can be added to the fleet without expanding existing infrastructure.

\section{Smart cars}

Path planning for autonomous cars involves determining the best route for the car to take from its current location to its destination while taking into account various factors such as traffic conditions, road conditions, and the car's capabilities. This process can be computationally intensive, especially if the car needs to consider the paths of multiple other cars in order to find the optimal route. In such cases, it may be beneficial to use cloud computing resources to help handle the increased complexity. By upstreaming some of the computations to the cloud, the autonomous car can offload some of the workloads and potentially make better use of its local computing resources. Additionally, using multi-car path planning can help to reduce traffic congestion and improve the overall efficiency of the road system, as the cars can coordinate their routes to avoid conflicts and optimize their use of available road space \cite{smart_cars}.