%!TEX root = ../Thesis.tex
\chapter{Introduction}

\begin{itemize}
    \item \textup{Upright shape}
    \item \textit{Italic shape}
    \item \textsl{Slanted shape}
    \item \textsc{Small Caps shape}
    \item \textmd{Medium series}
    \item \textbf{Bold sereies}
    \item \textrm{Roman family}
    \item \textsf{Sans serif family}
    \item \texttt{Typewriter family}
\end{itemize}


\begin{algorithm}
\caption{Modified mini-batch $K$-means} \label{modifiedminibatch}
\begin{algorithmic}[1]
\State Given: $K$, mini-batch size $B$, iterations $T$, dataset $X$, correlation~matrix~$\mathrm{P}$.
\State Initialize $C = \{\mathbf{c}^{(1)}, \mathbf{c}^{(2)}, \ldots, \mathbf{c}^{(K)}\}$ with random $\mathbf{x}$'es picked from $X$.
\State $A \gets B \cdot T$ sorted random indexes to $X$, denoted $a_1, a_2, \ldots, a_{B\cdot T}$.
\State $X' \gets \{\mathbf{x}^{(a_1)}, \mathbf{x}^{(a_2)}, \ldots, \mathbf{x}^{(a_{B\cdot T})}\}$ \Comment{Cache all points}
\State $\textbf{size} \gets 0$
\For {$i = 1$ to $T$}
    \State $M \gets B$ examples picked randomly from $X'$
    
    \For{$\mathbf{x} \in M$} \Comment{\textit{Assignment step}}
        \State $\textbf{d}[\textbf{x}] \gets f(C,\mathbf{x}, \mathrm{P})$ \Comment{Cache closest center}
    \EndFor
    
    \For {$\mathbf{x} \in M$} \Comment{\textit{Update step}}
        \State $\textbf{c} \gets \textbf{d[x]}$ \Comment{Get cached center for current \textbf{x}}
        \State $\textbf{size}[\textbf{c}] \gets \textbf{size}[\textbf{c}] + 1$ \Comment{Update cluster size}
        \State $\eta \gets \frac{1}{\textbf{size}[\textbf{c}]}$       \Comment{Get learning rate}
        \State $\textbf{c} \gets (1 - \eta)\textbf{c}+\eta\textbf{x}$ \Comment{Take gradient step}
    \EndFor
\EndFor
\State \Return {$C$, \textbf{size}}
\end{algorithmic}
\end{algorithm}


\begin{adjustwidth*}{0cm}{-0.4cm}
\begin{lstlisting}[language=Python,caption=Fibonacci2,label=Fibonacci2]
# This is a comment
import easy
str = "I am a string"
str2 = "Now i have an awsome string with ´ '' `` which are not TeX'ed"
str3 = "What about awsome unicode characters? Like “, π, ”, Ω, ç. \" This"
def fib(n):
    if n == 0:
        return 0
    elif n == 1:
        return 1
    else:
        return fib(n-1) + fib(n-2)
str4 = "Yes it is possible with 80 charactes. Which this string proves. Wiiii."
str5 = "It adjusts according to the spine"
\end{lstlisting}
\end{adjustwidth*}