%!TEX root = ../Thesis.tex
In this thesis, the problem of multi-agent path planning was discussed. A system that supports collaborative and cloud-enhanced path planning was designed, implemented, and tested using agents simulated on multiple devices. Part of the system representing local agents and their environment was deployed on physical devices, and another part responsible for visualization and delegation of computation was deployed in the public cloud in a way that in can support multiple tenants. Dijkstra, A*, and CA* algorithms were implemented and used both in local and cloud computation.

Twenty-four-hour latency test(figure \ref{fig:on_prem_test_time}) was performed to determine the average latency between local agents and the cloud system. On purpose, the location of the cloud system(Central USA) was selected to be distant from the location of the agents located in Copenhagen, Denmark to investigate the influence of network latency on performance. The average latency for this test was 131ms. The largest deviation was observed between 1 and 4 AM UTC (8-11 PM EST), which is a time when network lines in the USA might be highly utilized. This can be attributed to the fact that during these hours, many individuals and businesses in the USA are likely to be using the internet for various activities, leading to increased traffic on the network and resulting in higher latency. 

For testing purposes, an automatic test suite was created, which was testing the performance of all algorithms both in the local setup as well as in the cloud setup with 3 agents deployed on local devices. A* algorithm has significantly lower complexity  than CA* and therefore for agents with average resources(0.5 CPU, 512Mb memory), it is better to perform it locally. Given the largest map in the testing suite (961 tiles) it performed more than 3x faster than its cloud version. It is caused by a significant difference between network latency and algorithm execution time. Network latency is not dependent on algorithm complexity nor map size and therefore when map size increases the difference between the cloud algorithm and local algorithm would become less significant.

Test performed for the CA* algorithm(figure \ref{fig:on_prem_test_time}) has shown that for agents with average resources, it is better to perform computation in the cloud. There was only one case, for the smallest map where CA* performed better locally. As CA* is more complex than A*, it needs more resources. On average cloud solution was around ten times faster than the local solution. It is caused by significantly higher resources of cloud-agent and insignificant network delay in comparison to algorithm compute time. Additionally, it was discovered that in all test cases, there were no network disturbances and one round trip was enough to perform computation in the cloud and deliver results.

Agents with different resource profiles(table \ref{tab:resorces_profile}) were tested to discover how the system behaves, given diverse hardware limitations(figure \ref{fig:on_prem_test_time_log}). The only agent which was able to find all possible using CA* algorithm performed locally, given a 10-second computation limit was an agent with the highest resource profile. On the contrary, agents using cloud-enhanced algorithms found all possible paths for every resource profile. For all resource profiles, the A* algorithm was faster to perform locally, however, for agents with the lowest resource profile results for both A* and Cloud-enhanced A* are comparable. Cloud  solution was 39\% slower on average despite being faster on maps larger than 23x23 tiles. It is caused by the throttling of local agent resources on larger maps, which is causing slow down in computation time. For all resources, profiles CA* performed significantly better in the cloud than in the local setup, which is an effect of high resource consumption for algorithms with high complexity.

The overall built system proved that given agents with hardware limitations it is usually beneficial to outsource computation to the cloud, even though the location of the cloud servers is distant from the agents. Even for simpler, non-collaborative algorithms, such as A* delay of around 150ms might be acceptable in an industry scenario. CA* algorithm proved to be computationally expensive and in most cases not feasible to be performed on a local agent machine with limited resources, however when computed in collaboration with the cloud provides sufficient performance.
