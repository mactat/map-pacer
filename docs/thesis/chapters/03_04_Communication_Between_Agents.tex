The broker is a critical component in the proposed system design, responsible for facilitating communication between microservices. In an on-premises system, the broker is deployed to allow communication between agents and map service. In an industry setup, it would typically be deployed on a stand-alone server located in the same subnet as the agents. It acts as a middleman, receiving all messages and redirecting them to the appropriate parties involved in the communication process. One broker can be used to support multiple systems, as long as they are connected to the same subnet as the broker. This is made possible by the messaging scheme, which is distinctive for every system (as described in section \ref{sec:app_02}), allowing the broker to correctly route the request. The broker is an essential component of the system, ensuring that all mobile robots can communicate effectively and enabling the system to function as intended.

The MQTT protocol was selected for communication between the agents because it is extremely lightweight and suitable for Internet of Things communication. This makes it an ideal choice for supporting agents with very limited resources, such as those with limited memory, CPU, or bandwidth. This protocol is designed to be efficient in terms of both bandwidth and power usage, which makes it well-suited for use in a large-scale, distributed system like the one proposed here.