%!TEX root = ../Thesis.tex
\chapter{System Design}
This section provides an overview of the system design for the development of a multi-agent system. The system design involves creating multiple instances of identical entities, simulating a real-world scenario where these entities would be physical robots connected to each other through some form of networking. To achieve this simulation, a microservices approach is employed, allowing for easy separation of agents, even on a single computer setup, and enabling the setup of constraints on computing power and memory usage.

The system design also incorporates the use of cloud and edge computing, with cloud computing being utilized for data gathering, computation, and visualization, and edge computing is utilized to model the behavior of agents. Communication and collaboration among the agents are facilitated through the implementation of mechanisms such as leader election, liveness checks, and local discovery. Additionally, this section will cover the implementation of data gathering and visualization and its integration into the overall system.

Python is chosen as the primary programming language for the implementation of this system. Python is a widely-used, high-level programming language that is known for its simplicity, readability, and flexibility \cite{python_docs}. Moreover, its extensive support for concurrency and parallelism allows for efficient and effective handling of large amounts of messages, which is crucial for the multi-agent system.
\section{Microservices}
In a recent years wide cloud adoption has coused huge momentum in adopting software architecture known as micro-services, as it is a native architecture for the cloud. This project will also adapt this architecture because part of the deployment is being done in cloud, and swarm of agents can also be simulated using this approach. Moreover technologies such as kind and tilt made it easier to develop all services locally using local kubernetes cluster.

{\color{red}What are microservices?}

\begin{figure}[H]
    \centering
    \includegraphics[width=\textwidth]{pictures/services.png}
    \caption{ Micro services system design }
    \label{fig:micro_services}
\end{figure}

\subsection{Agent}
{\color{red}What time should i use(will, is, was)?}
Agent is a microservice which is implementing the logic of the entity which needs to plan a path.  It is meant to have limited resources to simulate real world scenario, and number of its instances will vary.

When agent is started, it doesn't belong to any swarm so it will start local discovery to find his peers. After peers are found, one of the agent has to be elected as a leader and therefore election algorithm will be initiated in on e or multiple agent and on agent will be elected to be a swarm leader. Agent will periodically poll his peers to check their liveness and it this behaviour will be refereed as liveness check.

This entity contain also all algorithms meant for path planning as in a base case scenario, computation will take place in the agent to obtain path.

\subsection{Cloud-agent}
Cloud-agent is an entity to which computation can be upstream to offload the agents. It will have much more computing power, but it might be located in different geographically location so latency between local agents and this one will matter in case of total calculation time.

This microservice will be deployed in the cloud and it will use cloud-broker as a communication medium.

\subsection{Broker}
Service responsible of communication between the services. It will be deployed both in cloud and in the local environment, and those two entities should be connected with a bridge. This will be a single connection point between cloud and edge.

{\color{red}Maybe explain in mqtt part)?}
This entity will use pub/sub approach to handle and distribute the messages.

\subsection{Map-service}
Initially when agents are spawned they do not have any map assigned. After electing a leader he will trigger new map generation, by message passing to map-service. This entity is responsible of creation the map and spawning agent in it. It is a substitute of sensing for the agent as it gives the robots all information they need for finding in position in a global map.

Additionally pre-defined map are stored in this service and can be adopted by the agent upon specific query. Note that map-service can spawn agent in a places wher it is not possible to reach the goal, but it is also a valid scenario.

\subsection{Backend}
It is a service meant for gathering data and interacting with a core module(agent, map-service etc.). It exposes endpoint throug REST api(endpoints in the appendix A) and forwards request to MQTT broker. It is also receiving data from alll the entities and stores it in a database connected to it. Backend is deployed in a cloud and it connects to a cloud-broker, endpoints should be exposed to the end-user.

\subsection{Frontend}
End user gragphical interface, deployed as a web-service is called frontend. It is taking inputs from the user and triggering specific actions through backend service API. It is also used for raw data visualization and it is deployed in a cloud and exposed to end user. It's capabilities are explained in \hyperref[sec:0308]{visualization section} of this chapter.

\section{Cloud Computing}
Goal of design of cloud deployments was to be cloud agnostic, meaning it should work with any cloud provider with minimal/no changes. For that reason all services are meant to be deployed in a Kubernetes cluster, but the location or structure of the cluster doesn't metter. All of the biggest cloud providers have Kubernetes as a Service (KaaS) offering moreover it is very common along smaller providers as well.

Overall design of the cloud services can be seen on figure \ref{fig:cloud_services}. Services are communicating with each other in two ways:
\begin{itemize}
    \item Via Cloud broker - Backend and Cloud-Agent are both subscribed to series of topics and exchanges information through MQTT messages sent to cloud-broker. 
    \item Via REST endpoints - Backend services exposes REST endpoint which is consumed by Frontend. Frontend itself is only meant to serve HTML documents and be a consumer of backend endpoint.
\end{itemize}

\begin{figure}[H]
    \centering
    \includegraphics[width=0.8\textwidth]{pictures/cloud_services.png}
    \caption{ Cloud services system design }
    \label{fig:cloud_services}
\end{figure}

Three endpoints are exposed to externally traffic(from outside of the cluster): two HTTP endpoints for Backend and Frontend and one MQTT enpoint for cloud-broker.
HTTP traffic is being served via ingress resources in the cluster it is forwarded via reverse proxy to kubernetes service which distributes the request among the pods. Default ingress resource is designed to only serve HTTP/HTTPS request and do not allow any other protocol. For MQTT protocol which is transported over TCP NodePort is used to expose service to outside world and map pod ports to ports of the node in a cluster.

Test

\section{Local Computing}
Agents, a local broker, and a map service are deployed on-premises and can facilitate the computation of the path in this setup. Agents can communicate with each other to agree on who will initiate the computation. Figure \ref{fig:local_services} shows how local services are deployed. Local broker and map services need to be accessible for all agents to communicate and map selection and therefore in most cases those services would be deployed on a stand-alone server in the same subnet as agents.

\begin{figure}[H]
    \centering
    \includegraphics[width=0.8\textwidth]{pictures/local_service.png}
    \caption{ Local services system design }
    \label{fig:local_services}
\end{figure}

This design can be extended, by deploying an edge agent on the same server as a map service and a broker. It would allow performing path computation with help of an edge agent which would have a higher resource profile than agents.

Agents can be deployed in a different way, depending on implementation. Preferably it should be a single binary package that could be deployed on machines with low resource profiles. All the agents need to have an ID or NAME assigned to them, and those have to be unique in the scope of the system. They should also be given information about the IP/URL of the local broker and a way to authenticate themselves. 

Map service and broker are more flexible in case of deployment because those services would probably be deployed on a stand-alone server with a high resource profile. recommended way would be to use a containers orchestrator such as solutions like Kubernetes or docker swarm and deploy the as containers/pods. It would assure the reliability and maintainability of those services. Otherwise, those can also be deployed on bare metal or in virtual machines.

\section{Communication Between Agents}
{\color{red}Write about all mqtt enpoints etc.}
Mqtt protocol was selected for communication between the agents, bacause(?). Broker services is the one who is facilitating the message exchange between all the services. Agents are subscribing to specific topics and receive a message every time when it is send to this topic. Hierarchy of topics is shown in appendix \ref{sec:app_02}.

\section{Leader Election}
Agents need to agree on multiple decisions, and therefore leader has to be elected among the agents. Leader will be responsible of generating and announcing the map, and up streaming data about the swarm to other services(backend). Main algorithm behind leader election is presented on the figure \ref{fig:election_logic}. 

\begin{figure}[H]
    \centering
    \includegraphics[width=0.8\textwidth]{pictures/election_logic.png}
    \caption{ Election logic}
    \label{fig:election_logic}
\end{figure}

Election will be triggered when new peer will be discovered or current leader is dead(not able to reach). Additional requirement is that there is no living leader already elected. After those checks are triggered, agent will send a message to other to start election, with random number signalizing importance of the election. Multiple elections can be started at the same time by multiple agents.

After starting an election, agent will be waiting to receive votes from all other agents, unless it will receive message about more important(higher number) election started. Once all the votes(randomly generated) are received agent with highest vote wins the election and will announce itself a leader, and other agents will accept it as a leader. To avoid the collision of both votes and  election importance number, those has to be set to large number.

\section{Map Design}
\label{sec:03_06_Map_Design}
This section focuses on the design of the map representation used in the multi-agent system. The map serves as a representation of the possible location of the agents in a discrete space. The initial map generated for the agents will indicate the starting and finishing points for each agent. Additionally, different algorithms may require the map to be translated into different forms, such as a graph. These different forms of representation are necessary to enable the efficient operation of specific algorithms. 

\subsection{Gridmap}
A grid map represents the potential locations of an agent in a discrete space. The starting and ending points for each agent are marked on the initial map, along with areas that are inaccessible, referred to as "walls" or "obstacles". Two map formats will be implemented: a space-separated map for creating and storing persistent maps and a human-readable format, and a JSON-based map for machine-to-machine communication and ease of parsing. The map service will handle the conversion and storage of maps, and will serve them to the agents, including the ability to generate random maps and place agents on them.

\begin{figure}[H]
    \centering
    \includegraphics[width=\textwidth]{pictures/map_2d.png}
    \caption{2D map to graph translation}
    \label{fig:map_2D}
\end{figure}

Each tile represents the node and it is connected with two edges to neighboring tiles, which represents the possibility of two-way movement from and to neighboring tiles. 


\subsection{Gridmap for CA* algorithm}
For CA* algorithm map has to be translated into a three dimensional(3D) grid map where every layer is a representation of the possible location of the agent in a particular time frame and therefore time becomes the third dimension in a graph. Similar to in a 2d grid map there are two possible modes which are represented in the figure below. An agent can only move in one direction vertically as each level corresponds to one moment in time. Other tiles on each level are connected in the same way as in figure \ref{fig:map_2D}.
\begin{figure}[H]
    \centering
    \includegraphics[width=\textwidth]{pictures/map_3D.png}
    \caption{3D map to graph translation}
    \label{fig:map_3D}
\end{figure}
After the path is planned for specific agents, it has to be marked as an obstacle so agents which are planning their paths, later on, would be aware of unreachable/occupied tiles.

\subsection{CA* front collision problem}
Planning multiple agents' paths on a single map leads to a problem when agents are colliding with each other because they are moving in the opposite direction on the same column or row. This situation is shown in Figure \ref{fig:head_collision}. Both agents assume that the tile in front of them is not occupied, which is a valid assumption. However, as this is a discrete model, a collision occurs between timeframes. To mitigate this problem, after the path is planned edges in the opposite direction of agent movement need to be removed.
\begin{figure}[H]
    \centering
    \includegraphics[width=0.7\textwidth]{pictures/head_collision_problem.png}
    \caption{CA* head collision problem}
    \label{fig:head_collision}
\end{figure}

\section{Liveliness Algorithm}
Map 2d is a representation of possible location of the agent.
\begin{figure}[H]
    \centering
    \includegraphics[width=0.8\textwidth]{pictures/agent_ttl.png}
    \caption{ Liveness check }
    \label{fig:liveness_check}
\end{figure}


\section{Visualization}
\label{sec:0308}
Maps and paths will be visualized in form of web application. It should be accessible from internet and connect to specific system to fetch live data. Mock of the visualization is shown on the figure \ref{fig:vis_mock}. 

\begin{figure}[H]
    \centering
    \includegraphics[width=\textwidth]{pictures/frontenf_mock.png}
    \caption{ Mock of the visualization }
    \label{fig:vis_mock}
\end{figure}

Graphical interface will be centered around gridmap which indicates positions and goal of all the agent(marked as coloured dots). On a side of the map will be status of the agent, indication of current leader and some details about result of the algorithm.

Through webapp, user will be able to(core functionality):
\begin{itemize}
\itemsep0em 
    \item Generate new map(random or predefined map)
    \item Create new map
    \item Select algorithm
    \item Visualize algorithm
\end{itemize}
Additional features:
\begin{itemize}
\itemsep0em 
    \item Access map creator, to specify own map.
    \item Choose among different systems(multi tenancy).
    \item Reply visualization.
\end{itemize}

\section{Multi tenancy}
The cloud-based component of this solution is designed to support multi-tenancy and multiple systems, utilizing shared cloud resources. The main principle of the multi-tenant system design is illustrated in Figure \ref{fig:multi_tenant_simple}. One tenant, for example, one factory, will be able to have multiple systems on-site, where a system refers to a group of agents, a map-service, and a broker. To optimize resource utilization, systems will also be able to share the broker, as detailed in Appendix \ref{sec:app_03}. This approach allows for efficient and effective management of the cloud resources while providing flexibility for individual tenants to have multiple systems on-site.

\begin{figure}[H]
    \centering
    \includegraphics[width=\textwidth]{pictures/multi_tenant_simple.png}
    \caption{ Multi-tenant system design }
    \label{fig:multi_tenant_simple}
\end{figure}

The cloud components of the solution are designed to be shared among tenants and accessed through a common web-based application. The final product will include a logging page where tenants can provide credentials to authenticate themselves. Once successfully logged in, tenants will have the option to choose the system that will be visualized.

The cloud services are designed to automatically scale as more tenants are onboarded. Additionally, the microservices are implemented as stateless, with the exception of the database where all tenant data is stored. Stateless services can be easily scaled horizontally through the use of auto-scaling, providing the system with greater scalability and flexibility. This approach allows the system to accommodate an increasing number of tenants while maintaining performance and stability.