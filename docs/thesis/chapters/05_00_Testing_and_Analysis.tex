%!TEX root = ../Thesis.tex
\chapter{Testing and Analysis}
The system mentioned in this work is complex and manual testing is not a viable approach, therefore automatic testing was implemented. For this purpose, an additional microservice was introduced, called a performance test. It utilizes backend REST API to create reproducible test cases. It is also gathering the results and shows performance KPIs (Key Performance Indicators) for the system. An example of test maps can be seen in Appendix \ref{sec:app_04}.

\section{Test cases}
System test consist of 5 map of different sizes and complexity level. Each algorithm is tested with every map with the same positioning of the agents. Test time is defined as end end-to-end time, starting from test triggering in a backend and stopping when backend receives paths from all agents. More detailed information such as "test setup time", "communication time" and "algorithm calculation time" can be found in observability suite(grafana).

\subsection{System simulated in the cloud}
This system can be simulated both in a local environment as well as in cloud. Multiple agents can be created and connected to a system by spawning multiple kubernetes pods. This scenario is not taking into account communication delay as all containers are communicating in a local network(cluster network).

\subsection{Real system}
Small system composed of three external devices(rasperry pi) was created to showcase how real system would work. In each of devices agent container was deployed. On one of them there are map-servcie and broker containers. Scheme of the devices connection can be seen on diagram \ref{fig:cluster}.
\begin{figure}[H]
    \centering
    \includegraphics[width=0.8\textwidth]{pictures/cluster.png}
    \caption{ Real cluster }
    \label{fig:cluster}
\end{figure}

This setup resemble real-world scenario, with 3 robots(agents) and one server meant for deploying broker and communicating to the cloud. Agents has limited resources as Single Board Computer used for deployment only has 1GB of memory and Quad core Cortex-A72 1.5GHz processor\cite{rpi_specs}. For the purpose of the test computing power can also be throttled. Devices are connected to local network via internet switch. 

Challenging part was that those devices are using ARM architecture, therefore containers has to be rebuild to support this architecture. To ease a development struggle, azure devops agents were installed on the machines and Continous Integration pipeline was put in place, which rebuilds ARM images on every code change using locl machines as hosts and pushes them into image registry.Cloud part of the solution is deployed in okteto cloud and both parts are exchanging messages via mqtt brokers.

\subsection{System composed from virtual machines}
Another way of testing the system is to create set of virtual machines or containers locally and connect them to cloud system. Benefits of this approach are that this setup is easier to maintain, it supports x86 architecture and is easily extensible, as for spawning more agent there is no additional hardware needed. Resource limits can also be implemented both in case of container and virtual machines. Those entities has to be put in same virtual network to enable communication between simulated machines.

\section{Communication time}
Real agents were located in Copenhagen, Denmark and cloud provider data center was based in city Council Bluffs, Iowa, United States. That positions them roughly 7500km apart from each other what might contribute to significant time delays.

24 hour latency test was conducted in order to gather more insight on expected latency between agents and the cloud. During the test ICMP package with 32 bytes of data was send every minute to the cloud broker endpoint and Round Trip Time(RTT) was measured. Results can be seen on diagram \ref{fig:ping}.

\begin{figure}[H]
    \centering
    \includegraphics[width=0.8\textwidth]{pictures/ping.png}
    \caption{ RTT time }
    \label{fig:ping}
\end{figure}

During 24h test average RTT was around 131ms, min 126ms, max 236ms, std 6ms over whole test. Largest standard deviation can be seen between 1 and 4 AM UTC.

\section{Comparison between computing time}
Test results for real agents were performed for agent devices forced to use half of their CPU and 512M of memory. Algorithms were grouped into non-collaborative group(A*, Cloud A*) and collaborative group(CA*, Cloud CA*), results can be seen on figures \ref{fig:on_prem_test_time} and in logarithmic scale on figure \ref{fig:on_prem_test_time_log}. Vertical axis represents sum of paths computing time in milliseconds and vertical axis indicates dimension of map i.ex map with dimension 5 will have 25(5x5) possible location for the agent.

\begin{figure}
    \centering
    \includegraphics[width=0.7\textwidth]{pictures/on_prem_test_time.png}
    \caption{Test on on-prem setup}
    \label{fig:on_prem_test_time}
\end{figure}
\begin{figure}
    \centering
    \includegraphics[width=0.7\textwidth]{pictures/on_prem_test_time_log.png}
    \caption{Test on on-prem setup in log scale}
    \label{fig:on_prem_test_time_log}
\end{figure}

Findings:
\begin{itemize}
    \item Non-collaborative algorithms are significantly faster than collaborative ones as they have lower time complexity. 
    \item Average time difference between A* and Cloud A* is around 150ms.
    \item Cloud A* has higher deviation of the results than A*.
    \item CA* beats Cloud A* on the smallest map, on any other map Cloud CA* is performing better.
    \item Cloud CA* is around 10x faster on average than CA*.
\end{itemize}

\section{Results by resources given to an agent}
There is a possibility to deploy an agent application on machines with different resource profiles and therefore it is important to determine how algorithms are performing given different resource limits.
In this test 4 agent profiles with limited CPU and memory were considered:
\begin{table}[H]
\label{tab:resorces_profile}
\centering
\begin{tabular}{@{}lll@{}}
\toprule
Profile name     & Memory limit & CPU limit \\ \midrule
very low profile & 128 MB        & 0.1 CPU   \\
low profile      & 128 MB        & 0.2 CPU   \\
medium profile   & 256 MB        & 0.3 CPU   \\
high profile     & 800 MB        & 1.0 CPU     \\ \bottomrule
\end{tabular}%
\end{table}

Memory limit refers to the maximum amount of memory that can be used before the process running application in the agent will be terminated. CPU limit will ensure that the agent can only use a specific part of the host CPU resources, i.e. setting it to 0.5 will produce the same output as running an application with 2 times slower CPU\cite{docker_container_limits}. A time limit of 10s was given to the agents while performing computations.

A comparison of agents running with different resource profiles according to the percentage of paths found can be seen in figure \ref{fig:compare_profiles_paths}. In the case of the first free algorithms(A*, Cloud A*, Cloud CA*), agents are able to find all possible paths which are equal to around 95\% of the total number of paths for all resources profile. 

In the case of the CA* algorithm, agents with very low profiles can only find 50\% of the total path given, low profile 65\%, medium profile - nearly 80\%, and high profile 95\%. 
\begin{figure}[H]
    \centering
    \includegraphics[width=0.8\textwidth]{pictures/compare_profiles_paths.png}
    \caption{Percentage of paths found by an agent with different resources profile
    }
    \label{fig:compare_profiles_paths}
\end{figure}

Figure \ref{fig:compare_profiles_log} shows the performance of agents with different resource profiles in relation to the average computing time for all agents. Time is given in milliseconds and shown on a logarithmic scale.

An agent with the low resource profile performed similarly for A* and cloud A* algorithms, however, Cloud CA* computing time was more than 10x smaller than local CA*. Further increase of agent resources, results in lowering of computing time. For agents with high resources, A* performed significantly better than cloud A*, and results from cloud CA* and CA* were similar, but cloud CA* performed better.
\begin{figure}[H]
    \centering
    \includegraphics[width=0.8\textwidth]{pictures/compare_profiles_log.png}
    \caption{Difference in compute time by agent resources}
    \label{fig:compare_profiles_log}
\end{figure}

