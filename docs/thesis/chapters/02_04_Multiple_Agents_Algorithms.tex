%!TEX root = ../Thesis.tex
Basic approaches such as A* solve the problem for single agents, but cannot be used in a multi-agent approach without modifications. The reason is that a multi-agent scenario is more dynamic and agents have to avoid collision with each other so it is not enough to just plan a simple route for every agent separately. Computing power is often an additional constraint, as an agent is usually a computer with limited resources. Therefore it is beneficial if agents are cooperating with each other. Most of the algorithms used in this setup are based on A* with tweaks that allow multi-agent cooperation\cite{path_adv}.

There are two main kinds of multi-agent path-finding: collaborative and non-collaborative. In the collaborative approach, it is assumed that all agents have full knowledge of all the others whereas in the non-collaborative it must predict the movement of the other agents as it has limited knowledge. This paper will mostly focus on the former.
\subsection{CA*}
The Cooperative A* algorithm is also divided into a smaller number of individual agent searches, but it takes into consideration the routes of other agents. It uses a 3-dimensional representation of a time-space map. Usually, 2 dimensions are storing the location of the agents and 3rd one represents the time frame, this structure is called the reservation table. The table is shared between the agents so they can mark the other agents as obstacles at any given time to avoid collision.\cite{path_adv}. 

The problem with this approach is that the routes have to be calculated in a sequence in 3-dimensional space which can slow down the execution of the algorithm and increase the level of coordination between agents as they have to agree on the sequence of calculation. Moreover, it might not be ideal for a cloud collaborative approach as even an agent with more computational power will not be able to solve the task more efficiently as it has to invoke computations in a sequence.

The algorithm works the same as the original A*, the only differences are the creation of the grid map which is now 3 dimensional, and  performing of the computation in sequence. Grid map has to be constructed in a way to disallow agents to move backward along the z-axis which symbolizes time. To achieve that nodes of graphs have to be connected only in one direction. More information about the structure of the map and the connection between the nodes can be found in section \ref{sec:03_06_Map_Design}.