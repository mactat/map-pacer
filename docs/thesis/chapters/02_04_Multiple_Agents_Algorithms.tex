%!TEX root = ../Thesis.tex
Basic approaches such as A* solves the problem for single agent, but cannot be used in a multi-agent approach without modifications. The reason is that multi-agent scenario is more dynamic and agents has to avoid collision with each other so it is not enough to just plan simple route for every agent separately. Computing power is often an additional constrain, as an agent is usually a computer with limited resources. Therefore it is beneficial if agents are cooperating with each other. Most of algorithms used in this setup are based on A* with tweaks that allows multi-agent cooperation f.e LRA* introduces reroute on demand by using "Local Repair"\cite{path_adv}.

There two main kinds multi-agent path-finding: collaborative and non-collaborative. In collaborative approach it is assumed that all agents have full knowledge of all the other where in non-collaborative it must predict movement of the other agents as it has limited knowledge. This paper will mostly focus on former.
\subsection{CA*}
Cooperative A* algorithm is also divided into smaller number of individual agent searches, but it takes into consideration routes of other agents. It uses 3 dimensional representation of a time-space map. Usually 2 dimensions are storing the location of the agents and 3rd one represents time frame, this structure is called reservation table. Table is shared between the agents so they can mark the other agents as obstacle in any given time to avoid the collision. Algorithm is not able to solve some problems, because first agents which are calculating their path, are not taking into consideration any other agent which can lead to blocking the path for next agent and render them unable to reach the goal.\cite{path_adv}. 

Problem with this approach is that the routes has to be calculated in a sequence in 3-dimensional space which can slow down execution of the algorithm and increase the level of coordination between agents as they have to agree on the sequence of calculation. Moreover it might not be ideal for cloud collaborative approach as even an agent with more computational power will not be able to solve the task more efficiently as it has to invoke computations in a sequence.