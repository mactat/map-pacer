%!TEX root = ../Thesis.tex
\chapter{Abstract}
In this master thesis, the problem of efficient path planning for autonomous navigation is addressed through the exploration of collaborative computing approaches. The goal of the project is to develop an end-to-end system for path planning that can be used with both simulated agents and real edge devices. In order to achieve this goal, the project investigates both collaborative computing between agents and edge-cloud collaborative computing. The focus of the project is not only on the implementation of specific algorithms but rather on the overall system design and performance. The research also assesses the impact of network latency on algorithm performance and the limits of agent resources that make it more beneficial to run algorithms in the cloud. An end-to-end system was designed and implemented, using simulated agents and physical edge devices, to test the performance of the A* and CA* algorithms in both local and cloud computation. Results showed that for agents with average resources, the A* algorithm performed better locally, while the CA* algorithm performed better in the cloud. The system was also tested with agents of varying resource profiles and computation limits, revealing the importance of matching algorithms to resource capabilities. Overall, the system demonstrated that, even for simple algorithms like A*, outsourcing computation to the cloud can be beneficial for agents with hardware limitations, and that the collaborative CA* algorithm provides sufficient performance only when computed in the cloud.

